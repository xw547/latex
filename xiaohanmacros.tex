\setlength{\oddsidemargin}{0.25 in}
\setlength{\evensidemargin}{0.25 in}
\setlength{\topmargin}{-0.6 in}
\setlength{\textwidth}{6.5 in}
\setlength{\textheight}{8.5 in}
\setlength{\headsep}{0.75 in}
\setlength{\parindent}{0 in}
\setlength{\parskip}{0.1 in}

%
% ADD PACKAGES here:
%

\usepackage{amsmath, amsfonts, graphicx, amsthm, amssymb, mathtools, bbm, units,             % Math typesetting packages (units for \nicefrac)
graphicx, wrapfig, caption, float,                            % Graphics packages
listings, color, inconsolata, pythonhighlight,               % Code packages
fancyhdr, sectsty, hyperref, enumerate, enumitem, framed,
setspace, array}
\newcolumntype{P}[1]{>{\centering\arraybackslash}p{#1}}

\hypersetup{colorlinks=true, linkcolor=magenta}


%
% The following commands set up the lecnum (lecture number)
% counter and make various numbering schemes work relative
% to the lecture number.
%
\newcounter{lecnum}
\renewcommand{\thepage}{\thelecnum-\arabic{page}}
\renewcommand{\thesection}{\thelecnum.\arabic{section}}
\renewcommand{\theequation}{\thelecnum.\arabic{equation}}
\renewcommand{\thefigure}{\thelecnum.\arabic{figure}}
\renewcommand{\thetable}{\thelecnum.\arabic{table}}

%
% The following macro is used to generate the header.
%
\newcommand{\lecture}[4]{
   \pagestyle{myheadings}
   \thispagestyle{plain}
   \newpage
   \setcounter{lecnum}{#1}
   \setcounter{page}{1}
   \noindent
   \begin{center}
   \framebox{
      \vbox{\vspace{2mm}
    \hbox to 6.28in { {\bf STSCI 6730 Mathematical Statistics
	\hfill Spring 2021} }
       \vspace{4mm}
       \hbox to 6.28in { {\Large \hfill Lecture #1: #2  \hfill} }
       \vspace{2mm}
       \hbox to 6.28in { {\it Lecturer: #3 \hfill Scribes: #4} }
      \vspace{2mm}}
   }
   \end{center}
   \markboth{Lecture #1: #2}{Lecture #1: #2}

   {\bf Note}: {\it LaTeX template courtesy of UC Berkeley EECS dept.}

   {\bf Disclaimer}: {\it These notes have not been subjected to the
   usual scrutiny reserved for formal publications.  They may be distributed
   outside this class only with the permission of the Instructor.}
   \vspace*{4mm}
}


% Use these for theorems, lemmas, proofs, etc.
\newtheorem{theorem}{Theorem}[lecnum]
\newtheorem{lemma}[theorem]{Lemma}
\newtheorem{proposition}[theorem]{Proposition}
\newtheorem{claim}[theorem]{Claim}
\newtheorem{corollary}[theorem]{Corollary}
\newtheorem{definition}[theorem]{Definition}
\newtheorem{example}[theorem]{Example}

% **** IF YOU WANT TO DEFINE ADDITIONAL MACROS FOR YOURSELF, PUT THEM HERE:

% -- Math/Statistics commands --

% 'numberthis' is a custom command that adds a reference number to a single line of a
% multi-line equation, e.g. "\numberthis\label{(name here)}" in align/gather environment
\newcommand\numberthis{\addtocounter{equation}{1}\tag{\theequation}}

% Shortcuts for bold math symbols; \bg{...} is for bold Greek letters
\let\b\mathbf
\let\bg\boldsymbol

% \mathscr{(letter here)} is a nice font for vector spaces
\usepackage[mathscr]{euscript}

% Convergence: right arrow with optional text above
% Usage: '\converge[w]' for weak convergence
\newcommand{\converge}[1][]{\xrightarrow{#1}}

% Normal distribution: first argument is the mean, second argument is the variance
% Usage: '\normal{\mu}{\sigma}'
\newcommand{\normal}[2]{\mathcal{N}(#1,#2)}

% Uniform distribution: first parameter is left endpoint, second parameter is right endpoint
% Usage: '\unif{0}{1}'
\newcommand{\unif}[2]{\text{Uniform}(#1,#2)}

% iid random variables
\newcommand{\iid}{\stackrel{\text{iid}}{\sim}}

% Natural, integer, rational, real, and complex numbers. Probability, expectation, and Borel sigma algebra. Non-italicized 'th' for math
% mode, e.g. $n^\tth$. Big-O.
\newcommand{\N}{\mathbb{N}}
\newcommand{\Z}{\mathbb{Z}}
\newcommand{\Q}{\mathbb{Q}}
\newcommand{\R}{\mathbb{R}}
\newcommand{\C}{\mathbb{C}}
\newcommand{\pr}{\mathbb{P}}
\newcommand{\E}{\mathbb{E}}
\newcommand{\B}{\mathcal{B}}
\newcommand{\tth}{\text{th}}
\newcommand{\Oh}{\mathcal{O}}
\newcommand{\sumni}[1]{\sum_{n={#1}}^{\infty}}
\newcommand{\sumn}{\sum_{n=0}^{\infty}}

%Declare a new environment using the current \enmerate enviroment.
\newenvironment{aenum}
{
\begin{enumerate}[label=(\alph*)]\setlength{\itemsep}{0pt}\setlength{\parskip}{0pt}
}{\end{enumerate}}

%Define a new enviroment aligned
\newenvironment{alieq}
{\begin{equation}
\begin{algined}}
{\end{aligned}
\end{equation}}


% argmax, argmin, argsup, arginf, variance, covariance, bias, range, 'with respect to', matrix diagonal, matrix trace
\DeclareMathOperator*{\argmax}{argmax}
\DeclareMathOperator*{\argmin}{argmin}
\DeclareMathOperator*{\argsup}{argsup}
\DeclareMathOperator*{\arginf}{arginf}
\DeclareMathOperator*{\var}{Var}
\DeclareMathOperator*{\cov}{Cov}
\DeclareMathOperator*{\bias}{Bias}
\DeclareMathOperator*{\ran}{ran}
\DeclareMathOperator*{\dv}{d\!}
\DeclareMathOperator*{\diag}{diag}
\DeclareMathOperator*{\trace}{trace}
\DeclareMathOperator*{\htheta}{\hat{\theta}}
\DeclareMathOperator*{\randvs}{X_1,\cdots,X_n}

